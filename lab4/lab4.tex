\documentclass{article}
\usepackage{fancyhdr}
\usepackage{ctex}
\usepackage{listings}
\usepackage{graphicx}
\usepackage[a4paper, body={18cm,22cm}]{geometry}
\usepackage{amsmath,amssymb,amstext,wasysym,enumerate,graphicx}
\usepackage{float,abstract,booktabs,indentfirst,amsmath}
\usepackage{array}
\usepackage{booktabs}
\usepackage{multirow}
\usepackage{url}
\usepackage{diagbox}
\renewcommand\arraystretch{1.4}
\usepackage{indentfirst}
\setlength{\parindent}{2em}
\usepackage{enumitem}
\setmonofont{DejaVu Sans Mono}
\usepackage{listings}
\usepackage{xcolor}
\usepackage{makecell}
\setCJKmonofont{黑体}
\usepackage{tikz}
\usepackage{tabularx}
\usetikzlibrary{positioning, arrows.meta}
\lstset{
    % language = C,
    xleftmargin = 3em,xrightmargin = 3em, aboveskip = 1em,
	backgroundcolor = \color{white}, % 背景色
	basicstyle = \small\ttfamily, % 基本样式 + 小号字体
	rulesepcolor= \color{gray}, % 代码块边框颜色
	breaklines = true, % 代码过长则换行
	numbers = left, % 行号在左侧显示
	numberstyle = \small, % 行号字体
    numbersep = -14pt, 
    keywordstyle=\color{purple}\bfseries, % 关键字颜色
    commentstyle =\color{red!50!green!50!blue!60}, % 注释颜色
    stringstyle = \color{red}, % 字符串颜色
    morekeywords={ASSERT, int64_t, uint32_t},
	frame = shadowbox, % 用(带影子效果)方框框住代码块
	showspaces = false, % 不显示空格
	columns = fixed, % 字间距固定
} 
\lstset{
    sensitive=true,
    moreemph={ASSERT, NULL}, emphstyle=\color{red}\bfseries,
    moreemph=[2]{int64_t, uint32_t, tid_t, uint8_t, int16_t, uint16_t, int32_t, size_t}, emphstyle=[2]\color{purple}\bfseries,
    }
%--------------------页眉--------------------%
\pagestyle{fancy}
\fancyhead[L]{}
\fancyhead[R]{}
\fancyhead[C]{华东师范大学软件工程学院实验报告}
\fancyfoot[C]{-\thepage-}
\renewcommand{\headrulewidth}{1.5pt}
%--------------------标题--------------------%
\begin{document}
\begin{center}
  \LARGE{{\textbf{\heiti 华东师范大学软件工程学院实验报告}}}
  \begin{table}[H]
    \centering
    \begin{tabular}{p{2cm}p{4cm}<{\centering}p{1cm}p{2cm}p{4cm}<{\centering}}
      实验课程:    & 计算机网络 & \quad & 年\qquad 级: & 2023级      \\ \cline{2-2} \cline{5-5}
      实验编号:    & Lab 04     & \quad & 实验名称:    & ARP
      \\ \cline{2-2} \cline{5-5}
      姓\qquad 名: & 王海生     & \quad & 学\qquad 号: & 10235101559 \\ \cline{2-2} \cline{5-5}
    \end{tabular}
  \end{table}
\end{center}
\rule{\textwidth}{1pt}
%--------------------正文--------------------%
\section{实验目的}
\begin{enumerate}[noitemsep, label={{\arabic*})}]
  \item 通过Wireshark获取ARP消息
  \item 掌握ARP数据包结构
  \item 掌握ARP数据包各字段的含义
  \item 了解ARP协议适用领域
\end{enumerate}
\section{实验内容与实验步骤}
\subsection{实验内容}


\subsubsection{捕获数据}

启动 \texttt{Wireshark},在菜单栏的捕获\(\to \)选项中进行设置,选择已连接的以太网,设置捕获过滤器为\texttt{arp},捕获\texttt{arp}数据包。

然后在命令行中使用\texttt{ipconfig -all}命令获取本机的\texttt{IP}地址和\texttt{MAC}地址。

在 \texttt{Wireshark} 的过滤器中输入\texttt{eth.addr==<yourMAC>
}(其中 \texttt{<yourMAC>}为本机的 \texttt{MAC}地址)。

在管理员模式下,使用 \texttt{arp -d} 命令清除本机的 \texttt{ARP} 缓存。

打开 \texttt{Wireshark},停止捕获。
\subsubsection{回答问题}

\begin{enumerate}[noitemsep]
	\item 通过语句“eth.addr==01:02:03:04:05:06”的形式,在wireshark中设置过滤器,找出与自己mac地址相关的arp报文。 Arp报文包括请求报文和应答报文,仔细分析两种报文的格式。
	\item 画出你的计算机和本地路由间ARP的请求和应答数据包,标记出请求和应答,为每个数据包 给出发送者和接受者的MAC和IP地址。
	\item
	\begin{enumerate}
		\item 画出你的计算机和本地路由间ARP的请求和应答数据包,标记出请求和应答,为每个数据包 给出发送者和接受者的MAC和IP地址。
		\item 什么样的操作码是用来表示一个请求?应答呢?
		\item 一个请求的ARP的报头有多大?应答呢?
		\item 对未知目标的MAC地址的请求是什么值?
		\item 什么以太网类型值说明ARP是更高一层的协议?
		\item ARP应答是广播吗?
	\end{enumerate}	
\end{enumerate}

\subsubsection{自主探索ARP报文}

去除过滤器,我们发现还有更多的arp报文。请研究这些额外的arp报文中,有什么其他的功能作用。

在查阅公开资料后,我将研究下列问题:

\begin{enumerate}[noitemsep]
	\item 其他计算机广播的ARP请求。本地网络上的其他计算机也在使用ARP。由于请求是以广播形式发送的,因此您的计算机将会接收到这些请求。
	\item 您的计算机发出的ARP回复。如果另一台计算机恰好对您的计算机的IP地址进行ARP查询,那么您的计算机将发送一个ARP回复以告知查询结果。
	\item 自发ARP(Gratuitous ARPs),其中您的计算机发送有关自身的请求或回复。当计算机或链接上线时,这有助于确保没有其他人正在使用相同的IP地址。自发ARP具有相同的发送方和目标IP地址,并且在Wireshark中它们的信息字段会标识其为自发ARP。
	\item 您的计算机发出的其他ARP请求及相应的ARP回复。在您清空其ARP缓存后,您的计算机可能需要对其他主机(不仅仅是默认网关)进行ARP查询。
\end{enumerate}



\subsection{实验步骤}

\begin{enumerate}[noitemsep, label={{\arabic*})}]
  \item 启动\texttt{Wireshark},在菜单栏的捕获\(\to \)选项中进行设置,选择已连接的以太网,设置捕获过滤器为\texttt{arp},将混杂模式设为关闭,然后开始捕获。
  \item 在命令行中使用\texttt{ipconfig -all}命令获取本机的\texttt{IP}地址和\texttt{MAC}地址。
        \begin{lstlisting}
    PS> ipconfig -all
  \end{lstlisting}
  \item 回到\texttt{Wireshark},设置捕获过滤器为 \texttt{eth.addr==<yourMAC>
        }
  \item 在管理员模式下,使用 \texttt{arp -d} 命令清除本机的 \texttt{ARP} 缓存。
        \begin{lstlisting}
    PS> arp -d
  \end{lstlisting}
  \item 打开 \texttt{Wireshark},停止捕获。
  \item 分析捕获到的\texttt{ARP}数据包,并回答相关问题。
  \item 对捕获到的\texttt{ARP}数据包进行数据分析,并回答相关问题。
  \item 问题讨论
\end{enumerate}

\section{实验环境}


\begin{itemize}[noitemsep]
  \item 操作系统:\texttt{Windows 11 家庭中文版 23H2 22631.2715}
  \item 网络适配器:\texttt{Killer(R) Wi-Fi 6 AX1650i 160MHz Wireless Network Adapter(201NGW)}
  \item \texttt{Wireshark}:\texttt{Version 4.2.0 (v4.2.0-0-g54eedfc63953)}
  \item \texttt{wget}:\texttt{GNU Wget 1.21.4 built on mingw32}
\end{itemize}


\section{实验结果与分析}

\subsection{捕获数据}

首先,打开\texttt{Wireshark},在菜单栏的捕获\(\to \)选项中进行设置,选择已连接的以太网,设置捕获过滤器为\texttt{arp},将混杂模式设为关闭,然后开始捕获。

\begin{figure}[H]
  \centering
  \includegraphics[width=0.8\textwidth]{img/1.png}
  \caption{设置捕获选项}
  \label{fig:1}
\end{figure}

然后在命令行中使用\texttt{ipconfig -all}命令获取本机的\texttt{IP}地址和\texttt{MAC}地址。

\begin{lstlisting}
	无线局域网适配器 WLAN:
	
	连接特定的 DNS 后缀 . . . . . . . :
	描述. . . . . . . . . . . . . . . : Killer (R) Wi-Fi 6 AX1650i 160MHz Wireless Network Adapter (201NGW)
	物理地址. . . . . . . . . . . . . : 10-3D-1C-CC-0F-D3
	DHCP 已启用 . . . . . . . . . . . : 是
	自动配置已启用. . . . . . . . . . : 是
	本地链接 IPv6 地址. . . . . . . . : fe80::d281:e59b:e8b4:5fe4%12 (首选)
	IPv4 地址 . . . . . . . . . . . . : 192.168.1.107 (首选)
	子网掩码  . . . . . . . . . . . . : 255.255.255.0
	获得租约的时间  . . . . . . . . . : 2024年12月8日 8:24:21
	租约过期的时间  . . . . . . . . . : 2024年12月8日 12:24:21
	默认网关. . . . . . . . . . . . . : 192.168.1.1
	DHCP 服务器 . . . . . . . . . . . : 192.168.1.1
	DHCPv6 IAID . . . . . . . . . . . : 135281948
	DHCPv6 客户端 DUID  . . . . . . . : 00-01-00-01-2B-EB-3E-08-F-C3-1C-9A-4C
	DNS 服务器  . . . . . . . . . . . : 180.168.255.118
	: 116.228.111.18
	TCPIP 上的 NetBIOS  . . . . . . . : 已启用
\end{lstlisting}

可以看到,本机的 \texttt{IP} 地址为 \texttt{192.168.1.107}, \texttt{MAC} 地址为 \texttt{10-3D-1C-CC-0F-D3}。

回到\texttt{Wireshark},设置捕获过滤器为 \texttt{eth.addr==10-3D-1C-CC-0F-D3}。

\begin{figure}[H]
  \centering
  \includegraphics[width=0.8\textwidth]{img/3.png}
  \caption{设置捕获过滤器}
  \label{fig:3}
\end{figure}

接下来,在管理员模式下,在终端中使用 \texttt{arp -d} 命令清除本机的 \texttt{ARP} 缓存。

\begin{figure}[H]
  \centering
  \includegraphics[width=0.3\textwidth]{img/4.png}
  \caption{清除本机\texttt{ARP}缓存}
  \label{fig:4}
\end{figure}

打开 \texttt{Wireshark},停止捕获。捕获结果如下图所示:

\begin{figure}[H]
  \centering
  \includegraphics[width=0.8\textwidth]{img/5.png}
  \caption{捕获结果}
  \label{fig:5}
\end{figure}

\subsection{回答问题}

\begin{enumerate}[noitemsep]
\item 通过语句“eth.addr==01:02:03:04:05:06”的形式,在wireshark中设置过滤器,找出与自己mac地址相关的arp报文。 Arp报文包括请求报文和应答报文,仔细分析两种报文的格式。

	\textbf{ARP请求报文格式:}
	\begin{center}
		\begin{tabular}{|c|c|c|}
			\hline
			\textbf{字段} & \textbf{长度(字节)} & \textbf{说明} \\
			\hline
			硬件类型 & 2 & 表示使用的数据链路层协议, 对于以太网为1 \\
			\hline
			协议类型 & 2 & 表示要映射的上层协议类型, 对于IPv4为0x0800 \\
			\hline
			硬件地址长度 & 1 & 硬件地址长度, 对于以太网MAC地址为6 \\
			\hline
			协议地址长度 & 1 & 协议地址长度, 对于IPv4地址为4 \\
			\hline
			操作代码 & 2 & ARP请求的操作代码为1 \\
			\hline
			发送方硬件地址 & 6 & 发送者的硬件地址(MAC地址)\\
			\hline
			发送方协议地址 & 4 & 发送者的协议地址(IP地址)\\
			\hline
			目标硬件地址 & 6 & 通常全为0,因为不知道目标的硬件地址 \\
			\hline
			目标协议地址 & 4 & 请求解析的目标协议地址(IP地址)\\
			\hline
		\end{tabular}
	\end{center}
	
	
	\textbf{ARP应答报文格式:}
	\begin{center}
		\begin{tabular}{|c|c|c|}
			\hline
			\textbf{字段} & \textbf{长度(字节)} & \textbf{说明} \\
			\hline
			硬件类型 & 2 & 表示使用的数据链路层协议, 对于以太网为1 \\
			\hline
			协议类型 & 2 & 表示要映射的上层协议类型, 对于IPv4为0x0800 \\
			\hline
			硬件地址长度 & 1 & 硬件地址长度, 对于以太网MAC地址为6 \\
			\hline
			协议地址长度 & 1 & 协议地址长度, 对于IPv4地址为4 \\
			\hline
			操作代码 & 2 & ARP应答的操作代码为2 \\
			\hline
			发送方硬件地址 & 6 & 响应者的硬件地址(MAC地址)\\
			\hline
			发送方协议地址 & 4 & 响应者的协议地址(IP地址)\\
			\hline
			目标硬件地址 & 6 & 初始请求中的发送方硬件地址 \\
			\hline
			目标协议地址 & 4 & 初始请求中的发送方协议地址 \\
			\hline
		\end{tabular}
	\end{center}
	
	\item 
	画出你的计算机和本地路由间ARP的请求和应答数据包,标记出请求和应答,为每个数据包 给出发送者和接受者的MAC和IP地址。
	
	选择 \texttt{ARP} 请求数据包,如下图所示:
	
	\begin{figure}[H]
		\centering
		\includegraphics[width=0.8\textwidth]{img/6.png}
		\caption{选择\texttt{ARP}请求数据包}
		\label{fig:6}
	\end{figure}
	
	可以看到,它包括了一个长度为 28 字节的 \texttt{ARP} 报头,其中包括了以下字段:
	
	\begin{itemize}[noitemsep]
		\item Hardware type: Ethernet (1),长度为 2 字节
		\item Protocol type: \texttt{IPv4}(0x0800),长度为 2 字节
		\item Hardware size: 6,长度为 1 字节
		\item Protocol size: 4,长度为 1 字节
		\item Opcode: request (1),长度为 2 字节
		\item Sender MAC address: \texttt{10:3d:1c:cc:0f:d3},长度为 6 字节
		\item Sender IP address: \texttt{192.168.1.107},长度为 4 字节
		\item Target MAC address: \texttt{00:00:00:00:00:00},长度为 6 字节
		\item Target IP address: \texttt{192.168.1.1},长度为 4 字节
	\end{itemize}
	
	画出 \texttt{ARP} 请求数据包,如下图所示:
	
	% 10 列的表格
	\begin{table}[H]
		\centering
		\begin{tabularx}{0.78\textwidth}{|*{10}{X|}}
			\hline
			\multicolumn{2}{|c|}{Hardware type}      & \multicolumn{2}{c|}{Protocol type}     & \multicolumn{1}{c|}{Hardware size} & \multicolumn{1}{c|}{Protocol size} & \multicolumn{2}{c|}{Opcode} & \multicolumn{2}{c|}{} \\
			\multicolumn{2}{|c|}{1}                  & \multicolumn{2}{c|}{0x0800}            & \multicolumn{1}{c|}{6}             & \multicolumn{1}{c|}{4}             & \multicolumn{2}{c|}{1}      & \multicolumn{2}{c|}{} \\
			\hline
			\multicolumn{6}{|c|}{Sender MAC address} & \multicolumn{4}{c|}{Sender IP address}                                                                                                                                 \\
			\multicolumn{6}{|c|}{10:3d:1c:cc:0f:d3}  & \multicolumn{4}{c|}{192.168.1.107}                                                                                                                                     \\
			\hline
			\multicolumn{6}{|c|}{Target MAC address} & \multicolumn{4}{c|}{Target IP address}                                                                                                                                 \\
			\multicolumn{6}{|c|}{00:00:00:00:00:00}  & \multicolumn{4}{c|}{192.168.1.1}                                                                                                                                       \\
			\hline
		\end{tabularx}
		\caption{\texttt{ARP}请求数据包}
	\end{table}
	
	选择一个 \texttt{ARP} 应答数据包,如下图所示:
	
	\begin{figure}[H]
		\centering
		\includegraphics[width=0.8\textwidth]{img/7.png}
		\caption{选择\texttt{ARP}应答数据包}
		\label{fig:7}
	\end{figure}
	
	画出 \texttt{ARP} 应答数据包,如下图所示:
	
	% 10 列的表格
	\begin{table}[H]
		\centering
		\begin{tabularx}{0.78\textwidth}{|*{10}{X|}}
			\hline
			\multicolumn{2}{|c|}{Hardware type}      & \multicolumn{2}{c|}{Protocol type}     & \multicolumn{1}{c|}{Hardware size} & \multicolumn{1}{c|}{Protocol size} & \multicolumn{2}{c|}{Opcode} & \multicolumn{2}{c|}{} \\
			\multicolumn{2}{|c|}{1}                  & \multicolumn{2}{c|}{0x0800}            & \multicolumn{1}{c|}{6}             & \multicolumn{1}{c|}{4}             & \multicolumn{2}{c|}{2}      & \multicolumn{2}{c|}{} \\
			\hline
			\multicolumn{6}{|c|}{Sender MAC address} & \multicolumn{4}{c|}{Sender IP address}                                                                                                                                 \\
			\multicolumn{6}{|c|}{58:41:20:b0:c3:87}  & \multicolumn{4}{c|}{192.168.1.1}                                                                                                                                       \\
			\hline
			\multicolumn{6}{|c|}{Target MAC address} & \multicolumn{4}{c|}{Target IP address}                                                                                                                                 \\
			\multicolumn{6}{|c|}{10:3d:1c:cc:0f:d3}  & \multicolumn{4}{c|}{192.168.1.107}                                                                                                                                     \\
			\hline
		\end{tabularx}
		\caption{\texttt{ARP}应答数据包}
	\end{table}
	
	\item
	\begin{enumerate}
		
		
		\item 什么样的操作码是用来表示一个请求?应答呢?
		
		\texttt{ARP} 报头中的 \texttt{Opcode} 字段用来表示 \texttt{ARP} 请求或应答,其中 \texttt{Opcode} 为 1 表示请求,为 2 表示应答。
		
		\item 一个请求的ARP的报头有多大?应答呢?
		
		长度均为 28 字节。
		\item 对未知目标的MAC地址的请求是什么值?
		
		对未知目标的 \texttt{MAC} 地址的请求是 \texttt{00:00:00:00:00:00}。
		\item 什么以太网类型值说明ARP是更高一层的协议?
		
		以太网类型值为 \texttt{0x0806} 说明 \texttt{ARP} 是更高一层的协议。
		
		\begin{figure}[H]
			\centering
			\includegraphics[width=0.8\textwidth]{img/8.png}
			\caption{\texttt{ARP} 的类型值为 \texttt{0x0806}}
			\label{fig:8}
		\end{figure}
		\item ARP应答是广播吗?
		
		在以太网层可以看出,\texttt{ARP} 应答是单播。
		
		\begin{figure}[H]
			\centering
			\includegraphics[width=0.8\textwidth]{img/9.png}
			\caption{\texttt{ARP} 应答是单播}
			\label{fig:9}
		\end{figure}
	\end{enumerate}
	
\end{enumerate}

\subsection{自主探索ARP报文}

\begin{enumerate}[noitemsep]
\item 其他计算机广播的ARP请求。本地网络上的其他计算机也在使用ARP。由于请求是以广播形式发送的,因此您的计算机将会接收到这些请求。

答:清除筛选后,可以看到其他计算机发送的 \texttt{ARP} 请求。

\begin{figure}[H]
  \centering
  \includegraphics[width=0.8\textwidth]{img/10.png}
  \caption{其他计算机发送的 \texttt{ARP} 请求}
  \label{fig:10}
\end{figure}

\item 您的计算机发出的ARP回复。如果另一台计算机恰好对您的计算机的IP地址进行ARP查询,那么您的计算机将发送一个ARP回复以告知查询结果。

答:可以在另一台计算机上使用 \texttt{arp -d <Your IP>} 命令清除 \texttt{ARP} 缓存,然后使用 \texttt{ping <Your IP>} 命令向本机发送 \texttt{ICMP} 请求,这时也会发起一个 \texttt{ARP}请求,此时本机会发送 \texttt{ARP} 应答,如下图所示(由于从 \texttt{WIFI} 换到了以太网, \texttt{IP} 地址发生了变化):

\begin{figure}[H]
  \centering
  \includegraphics[width=0.8\textwidth]{img/11.png}
  \caption{本机发送了 \texttt{ARP} 应答}
  \label{fig:11}
\end{figure}

\item 自发ARP(Gratuitous ARPs),其中您的计算机发送有关自身的请求或回复。当计算机或链接上线时,这有助于确保没有其他人正在使用相同的IP地址。自发ARP具有相同的发送方和目标IP地址,并且在Wireshark中它们的信息字段会标识其为自发ARP。

答:可以在捕获列表中看到 \texttt{gratuitous ARP} 数据包。

\begin{figure}[H]
\centering
\includegraphics[width=0.8\textwidth]{img/13.png}
\caption{捕获到的 \texttt{gratuitous ARP} 数据包}
\label{fig:13}
\end{figure}
\item 您的计算机发出的其他ARP请求及相应的ARP回复。在您清空其ARP缓存后,您的计算机可能需要对其他主机(不仅仅是默认网关)进行ARP查询。

答:清除 \texttt{ARP} 缓存后,观察到了相关请求。

\begin{figure}[H]
  \centering
  \includegraphics[width=0.8\textwidth]{img/12.png}
  \caption{相关请求}
  \label{fig:12}
\end{figure}

\end{enumerate}


\section{实验结果总结}

本次实验通过 \texttt{Wireshark} 捕获了 \texttt{ARP} 数据包,并对其进行了分析,了解了 \texttt{ARP} 数据包的结构和各字段的含义,进一步增强了对 \texttt{ARP} 协议的理解。


\section{附录}

无

\end{document}