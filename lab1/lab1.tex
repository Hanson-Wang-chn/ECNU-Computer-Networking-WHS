\documentclass{article}
\usepackage{fancyhdr}
\usepackage{ctex}
\usepackage{listings}
\usepackage{graphicx}
\usepackage[a4paper, body={18cm,22cm}]{geometry}
\usepackage{amsmath,amssymb,amstext,wasysym,enumerate,graphicx}
\usepackage{float,abstract,booktabs,indentfirst,amsmath}
\usepackage{array}
\usepackage{booktabs}
\usepackage{multirow}
\usepackage{url}
\usepackage{diagbox}
\renewcommand\arraystretch{1.4}
\usepackage{indentfirst}
\setlength{\parindent}{2em}
\usepackage{enumitem}
\setmonofont{Consolas}
\usepackage{listings}
\usepackage{xcolor}
\usepackage{makecell}
\setCJKmonofont{黑体}
\lstset{
    % language = C,
    xleftmargin = 3em,xrightmargin = 3em, aboveskip = 1em,
	backgroundcolor = \color{white}, % 背景色
	basicstyle = \small\ttfamily, % 基本样式 + 小号字体
	rulesepcolor= \color{gray}, % 代码块边框颜色
	breaklines = true, % 代码过长则换行
	numbers = left, % 行号在左侧显示
	numberstyle = \small, % 行号字体
    numbersep = -14pt, 
    keywordstyle=\color{purple}\bfseries, % 关键字颜色
    commentstyle =\color{red!50!green!50!blue!60}, % 注释颜色
    stringstyle = \color{red}, % 字符串颜色
    morekeywords={ASSERT, int64_t, uint32_t},
	frame = shadowbox, % 用(带影子效果)方框框住代码块
	showspaces = false, % 不显示空格
	columns = fixed, % 字间距固定
} 
\lstset{
    sensitive=true,
    moreemph={ASSERT, NULL}, emphstyle=\color{red}\bfseries,
    moreemph=[2]{int64_t, uint32_t, tid_t, uint8_t, int16_t, uint16_t, int32_t, size_t}, emphstyle=[2]\color{purple}\bfseries,
    }
%--------------------页眉--------------------%
\pagestyle{fancy}
\fancyhead[L]{}
\fancyhead[R]{}
\fancyhead[C]{华东师范大学软件工程学院实验报告}
\fancyfoot[C]{-\thepage-}
\renewcommand{\headrulewidth}{1.5pt}
%--------------------标题--------------------%
\begin{document}
\begin{center}
  \LARGE{{\textbf{\heiti 华东师范大学软件工程学院实验报告}}}
  \begin{table}[H]
    \centering
    \begin{tabular}{p{2cm}p{4cm}<{\centering}p{1cm}p{2cm}p{4cm}<{\centering}}
      实验课程:    & 计算机网络 & \quad & 年\qquad 级: & 2023级         \\ \cline{2-2} \cline{5-5}
      实验编号:    & Lab 01     & \quad & 实验名称:    & Protocol Layer
      \\ \cline{2-2} \cline{5-5}
      姓\qquad 名: & 王海生     & \quad & 学\qquad 号: & 10235101559    \\ \cline{2-2} \cline{5-5}
    \end{tabular}
  \end{table}
\end{center}
\rule{\textwidth}{1pt}
%--------------------正文--------------------%
\section{实验目的}
\begin{enumerate}[noitemsep, label={{\arabic*})}]
  \item 学会通过Wireshark获取各协议层的数据包
  \item 掌握协议层数据包结构
  \item 分析协议开销
\end{enumerate}
\section{实验内容与实验步骤}
\subsection{实验内容}


\subsubsection{获取协议层的数据包}
使用\texttt{wget}命令发起\texttt{HTTP}请求,然后使用\texttt{Wireshark}抓包。

\subsubsection{绘制数据包结构}

分析\texttt{HTTP} \texttt{GET}协议包的内容,并绘制协议包,分别标出\texttt{Ethernet},\texttt{IP}和\texttt{TCP}协议的头部的位置、大小以及其负载的范围。

\subsubsection{分析协议开销}

根据实验结构,分析\texttt{HTTP}应用协议额外开销。
估计上面捕获的\texttt{HTTP}协议的额外开销。
假设\texttt{HTTP}数据(头部和消息)是有用的,而\texttt{TCP},\texttt{IP}和\texttt{Ethernet}头部认为是开销。对于下载的主要部分中的每一个包,我们需要分析\texttt{Ethernet},\texttt{IP}和\texttt{TCP}的开销,和有用的\texttt{HTTP}数据的开销。根据以上的定义来估计下载协议的开销,你认为这种开销是必要的吗?

\subsubsection{分析解复用键}

\textbf{解复用}指找到正确的上一层协议来处理到达的包。

观察下载的以太网和\texttt{IP}包头部信息,回答下面问题:

\begin{enumerate}[noitemsep]
  \item 以太网头部中哪一部分是解复用(解复用: 找到正确的上一层协议来处理到达的包的行为叫做 解复用)键并且告知它的下一个高层指的是\texttt{IP},在这一包内哪一个值可以表示\texttt{IP}?
  \item \texttt{IP}头部中哪一部分是解复用键并且告知它的下一一个高层指的是TCP,在这一包内哪一个值可以表示\texttt{TCP}?
\end{enumerate}

\subsubsection{问题讨论}

\begin{enumerate}[noitemsep]
  \item 查看不包含高层数据的短TCP数据包,查看它发往哪?不携带高层数据的数据包有用吗?
  
  \item 在经典的分层模型中,低层字段包装到高层数据包外面,成为一条新消息。但这并非总是如此,Web响应(一个包含HTTP标头和HTTP有效负载的HTTP消息)可能被转换为多个较低层的消息(即多个TCP数据包)。假设你为Web响应的第一个和最后一个TCP数据包绘制了数据包结构,那么该结构与经典分层模型有什么不同?
  
  \item 在上述经典分层模型中,低层字段包装到高层数据包外面,如果较低层添加加密,此模型将如何更改?
  
  \item 在上述经典分层模型中,低层字段包装到高层数据包外面,如果较低的层添加压缩,此模型将如何更改?
  
\end{enumerate}


\subsection{实验步骤}

\begin{enumerate}[noitemsep, label={{\arabic*})}]
  \item 安装实验所需软件(为方便安装,我们使用\texttt{Windows}下的包管理器\texttt{winget})
        \begin{lstlisting}[language=bash]
    PS> winget install wget
    PS> winget install wireshark
    \end{lstlisting}
  \item 打开\texttt{Wireshark},在菜单栏的 \texttt{捕获 -> 选项} 中进行设置,选择已连接的网络,设置捕获过滤器为\texttt{tcp port 80},将混杂模式设为关闭,勾选 \texttt{enable network  name resolution},然后开始捕获。
  \item 使用 \texttt{wget} 命令发起 \texttt{HTTP} 请求
        \begin{lstlisting}[language=bash]
    PS> wget http://www.baidu.com
    \end{lstlisting}
  \item 在 \texttt{Wireshark} 中停止捕获。
  \item 分析\texttt{HTTP} \texttt{GET}协议包的内容,并绘制协议包。
  \item 分析协议开销
  \item 分析解复用键
  \item 问题讨论
\end{enumerate}

\section{实验环境}


\begin{itemize}[noitemsep]
  \item 操作系统:\texttt{Windows 11 家庭中文版 23H2 22631.2715}
  \item 网络适配器:\texttt{Killer(R) Wi-Fi 6 AX1650i 160MHz Wireless Network \\ Adapter(201NGW)}
  \item \texttt{Wireshark}:\texttt{Version 4.2.0 (v4.2.0-0-g54eedfc63953)}
  \item \texttt{wget}:\texttt{GNU Wget 1.21.4 built on mingw32}
\end{itemize}


\section{实验过程与分析}

\subsection{获取协议层的数据包}

\textbf{问题:根据抓取的HTTP请求的GET方法的抓取结果,分析协议包的内容。}

首先,我们打开\texttt{Wireshark},在菜单栏的 \texttt{捕获 -> 选项} 中进行设置,选择已连接的网络,设置捕获过滤器为\texttt{tcp port 80},将混杂模式设为关闭,勾选 \texttt{enable network  name resolution},然后开始捕获。

\begin{figure}[H]
  \centering
  \includegraphics[width=15cm]{images/01.png}
  \caption{开始捕获}
\end{figure}

然后,我们使用 \texttt{wget} 命令对\texttt{http://www.baidu.com}发起 \texttt{HTTP} 请求

\begin{figure}[H]
  \centering
  \includegraphics[width=15cm]{images/02.png}
  \caption{使用 \texttt{wget} 命令发起 \texttt{HTTP} 请求}
\end{figure}

最后,我们在 \texttt{Wireshark} 中停止捕获,得到如下结果:

\begin{figure}[H]
  \centering
  \includegraphics[width=15cm]{images/03.png}
  \caption{捕获结果}
\end{figure}

\subsection{绘制数据包结构}

\textbf{问题:画一个关于使用GET方法的HTTP请求的图(与下图类似),为了显示协议层的嵌套结构,请分别标出Ethernet, IP和TCP协议的头部的位置、大小以及其负载的范围。}

接下来,我们分析数据包的结构。在\texttt{Wireshark}中,我们选择\texttt{HTTP} \texttt{GET}协议包,可以看到其内容,如下图所示:

\begin{figure}[H]
  \centering
  \includegraphics[width=15cm]{images/04.png}
  \caption{\texttt{HTTP} \texttt{GET}协议包}
\end{figure}

其中,第一个块是\texttt{Frame},这不是一个协议,而是一个记录,描述有关数据包的整体信息,包括捕获时间和长度等。我们可以看到,整个数据包的长度是\texttt{158}字节。
第二个块是\texttt{Ethernet II},这是以太网协议,可以看到这个标头的长度是\texttt{14}字节。
第三个块是\texttt{Internet Protocol Version 4},这是\texttt{IP}协议,可以看到这个标头的长度是\texttt{20}字节。
第四个块是\texttt{Transmission Control Protocol},这是\texttt{TCP}协议,可以看到这个标头的长度是\texttt{20}字节。
第五个块是\texttt{Hypertext Transfer Protocol},这是\texttt{HTTP}协议,可以看到这个标头的长度是\texttt{104}字节。

\begin{figure}[H]
  \centering
  \begin{minipage}[b]{0.45\textwidth}
    \includegraphics[width=\textwidth]{images/05.png}
    \caption{数据包的整体信息}
  \end{minipage}
  \hfill
  \begin{minipage}[b]{0.45\textwidth}
    \includegraphics[width=\textwidth]{images/06.png}
    \caption{\texttt{以太网}协议包}
  \end{minipage}
\end{figure}

\begin{figure}[H]
  \centering
  \begin{minipage}[b]{0.45\textwidth}
    \includegraphics[width=\textwidth]{images/07.png}
    \caption{\texttt{IP}协议包}
  \end{minipage}
  \hfill
  \begin{minipage}[b]{0.45\textwidth}
    \includegraphics[width=\textwidth]{images/08.png}
    \caption{\texttt{TCP}协议包}
  \end{minipage}
\end{figure}

\begin{figure}[H]
  \centering
  \begin{minipage}[b]{0.45\textwidth}
    \includegraphics[width=\textwidth]{images/09.png}
    \caption{\texttt{HTTP}协议包}
  \end{minipage}
\end{figure}

协议包的结构如下图所示:

\begin{figure}[H]
  \centering
  \includegraphics[width=15cm]{images/10.png}
  \caption{协议包的结构}
\end{figure}

\subsection{分析协议开销}

\textbf{问题:根据数据包的抓取结果,分析协议开销。}

\texttt{以太网} 的开销为

$$
  \frac{14}{158}=8.86\%
$$

\texttt{IP} 的开销为

$$
  \frac{20}{158}=12.66\%
$$

\texttt{TCP} 的开销为

$$
  \frac{20}{158}=12.66\%
$$

整个捕获过程(从\texttt{SYN ACK}开始,
到\texttt{HTTP}后的第一个\texttt{TCP}数据包结束)的开销为

$$
  \frac{14+20+20+ 66 + 54 \times 2}{158 + 66 + 54 \times 2}= 68.7\%
$$

有用的\texttt{HTTP}数据的开销为

$$
  \frac{104}{158 + 66 + 54 \times 2}= 31.3\%
$$

\begin{figure}[H]
  \centering
  \includegraphics[width=15cm]{images/11.png}
  \caption{整个捕获过程}
\end{figure}

\subsection{协议开销分析与必要性讨论}

\textbf{问题:估计协议的开销或者是协议开销占用下载字节的百分比。对于下载的主要部分中的每一个包,我们需要分析 Ethernet,IP和TCP的开销,和有用的HTTP数据的开销,你认为这种开销是必要的吗? (假设HTTP数据(头部和消息)是有用的,而TCP,IP和Ethernet头部认为是开销。)}

\subsubsection{协议开销估计}

根据以太网、IP、TCP等协议的标准文档,我们可以估计协议开销占用下载字节的百分比。具体如下:

\texttt{以太网}:以太网帧头部通常包含14字节(前导码1字节、目标MAC地址6字节、源MAC地址6字节、类型2字节),尾部包含4字节的FCS(帧校验序列)。因此,以太网的开销为:
$$
\frac{14 + 4}{158} = \frac{18}{158} = 11.39\%
$$

\texttt{IP}:IPv4头部通常为20字节(最小值),因此IP的开销为:
$$
\frac{20}{158} = 12.66\%
$$

\texttt{TCP}:TCP头部通常为20字节(最小值),因此TCP的开销为:
$$
\frac{20}{158} = 12.66\%
$$

综合来看,整个捕获过程中(从\texttt{SYN ACK}开始,到\texttt{HTTP}后的第一个\texttt{TCP}数据包结束),包括以太网、IP、TCP头部及后续的握手过程,总开销为:
$$
\frac{18 + 20 + 20 + 66 + 54 \times 2}{158 + 66 + 54 \times 2} = \frac{242}{386} = 62.69\%
$$

有用的数据,即\texttt{HTTP}数据的开销为:
$$
\frac{104}{158 + 66 + 54 \times 2} = \frac{104}{386} = 26.94\%
$$

\subsubsection{协议开销的必要性}

对于下载的主要部分中的每一个包,以太网、IP和TCP的头部信息确实被视为开销。然而,这些开销是确保数据能够正确无误地从发送方传输到接收方所必需的。具体来说:

\begin{enumerate}[noitemsep, label={{\arabic*})}]
	\item \texttt{以太网}头部包含了MAC地址等信息,用于局域网内的设备识别和寻址。
	\item \texttt{IP}头部则提供了互联网级别的路由信息,确保数据包可以跨多个网络进行传递。
	\item \texttt{TCP}头部负责建立连接、确认机制、流量控制等功能,保证了数据传输的可靠性。
\end{enumerate}

因此,虽然这些头部信息占用了部分传输容量,但它们的存在是实现有效、可靠通信的基础。没有这些协议头部提供的功能,数据传输将无法正常工作,网络服务也将变得不可靠。综上所述,尽管存在一定的开销,但这些开销是网络通信不可或缺的一部分,具有其必要性。

\subsection{分析解复用键}

\begin{enumerate}[noitemsep]
  \item \textbf{以太网头部中哪一部分是解复用(解复用: 找到正确的上一层协议来处理到达的包的行为叫做 解复用)键并且告知它的下一个高层指的是\texttt{IP},在这一包内哪一个值可以表示\texttt{IP}?}

  答:\texttt{以太网}头部中的\texttt{type}字段是解复用键,它的值为\texttt{0x0800},表示\texttt{IP}协议。

  \item \textbf{\texttt{IP}头部中哪一部分是解复用键并且告知它的下一一个高层指的是TCP,在这一包内哪一个值可以表示\texttt{TCP}?}

  答:\texttt{IP}头部中的\texttt{protocol}字段是解复用键,它的值为\texttt{0x06},表示\texttt{TCP}协议。
\end{enumerate}

\begin{figure}[H]
  \centering
  \includegraphics[width=13cm]{images/12.png}
  \caption{\texttt{以太网}头部}
\end{figure}

\begin{figure}[H]
  \centering
  \includegraphics[width=13cm]{images/13.png}
  \caption{\texttt{IP}头部}
\end{figure}

\subsection{问题与思考}

\begin{enumerate}[noitemsep]
  \item \textbf{查看不包含高层数据的短TCP数据包,查看它发往哪?不携带高层数据的数据包有用吗?}

  答:这个包的目的地是我们要访问的网站或者我们自己,通常是确认和控制信息类报文,用于建立\texttt{TCP}连接,对\texttt{HTTP}协议是有用的。这是\texttt{TCP}三次握手的一部分,用于建立\texttt{TCP}连接。

  \item \textbf{在经典的分层模型中,低层字段包装到高层数据包外面,成为一条新消息。但这并非总是如此,Web响应(一个包含HTTP标头和HTTP有效负载的HTTP消息)可能被转换为多个较低层的消息(即多个TCP数据包)。假设你为Web响应的第一个和最后一个TCP数据包绘制了数据包结构,那么该结构与经典分层模型有什么不同?}

  答:第一个\texttt{TCP}数据包的负载是\texttt{HTTP}协议的头部,而其余\texttt{TCP}数据包的负载是\texttt{HTTP}的有效载荷。


  \item \textbf{在上述经典分层模型中,低层字段包装到高层数据包外面,如果较低层添加加密,此模型将如何更改?}

  答:双方需要协商加密算法,然后在传输数据时,接收方需要解密数据,否则无法解析数据包。

  \item \textbf{在上述经典分层模型中,低层字段包装到高层数据包外面,如果较低的层添加压缩,此模型将如何更改?}

  答:发送方可以将解压缩方法附在头部,接收方在接收到数据包后,可以解压缩数据包,然后解析数据包。

\end{enumerate}

\section{实验结果总结}

通过本次实验,我学会了通过\texttt{Wireshark}获取各协议层的数据包,掌握了协议层数据包结构,分析了协议开销,分析了解复用键。对计算机网络中的协议层有了更深的理解。

\section{附录}

无

\end{document}